\documentclass{article}
\usepackage{ctex}
\usepackage{amsmath}
\usepackage{geometry}
\geometry{a4paper, margin=1in}

\title{部分映射交叉(PMX)介绍}
\author{}
\date{}

\begin{document}

\maketitle

\section*{引言}
部分映射交叉(PMX,Partially Mapped Crossover)是一种用于遗传算法的交叉操作方法,常用于解决组合优化问题,如旅行商问题(TSP)。PMX通过保持父代个体的一部分基因序列并映射其余部分,从而生成新的子代个体。以下是PMX的基本步骤:

\section*{步骤}

\subsection*{1. 选择交叉点}
随机选择两个交叉点,将染色体分为三个部分:左部分、中间部分和右部分。

\subsection*{2. 交换中间部分}
在两个父代个体中,交换这两个交叉点之间的中间部分,从而生成初步的子代个体。

\subsection*{3. 处理映射关系}
为了保证每个基因在子代个体中只出现一次,对中间部分的交换进行映射处理。如果一个基因已经存在于中间部分之外的区域,则通过映射关系进行替换。

\subsection*{4. 构建子代个体}
完成映射处理后,将所有部分组合起来,形成完整的子代个体。

\section*{举例说明}
假设我们有两个父代个体P1和P2:
\begin{align*}
\text{P1} &: [1, 2, 3, 4, 5, 6, 7, 8, 9] \\
\text{P2} &: [9, 8, 7, 6, 5, 4, 3, 2, 1]
\end{align*}

我们选择交叉点为位置3和7。交叉后,初步子代如下:
\begin{align*}
\text{子代1 (从P1和P2交叉)} &: [1, 2, 3, 6, 5, 4, 3, 8, 9] \\
\text{子代2 (从P2和P1交叉)} &: [9, 8, 7, 4, 5, 6, 7, 2, 1]
\end{align*}

然后,我们处理映射关系:
\begin{itemize}
    \item 对于子代1,从P2的中间部分(6, 5, 4, 3)替换到子代1的相应位置,确保每个基因唯一。此时的映射关系是:(6↔4), (5↔5), (4↔6), (3↔3)。
    \item 处理后,子代1可能变成: [1, 2, 3, 6, 5, 4, 7, 8, 9]
    \item 对子代2进行类似处理:
    \item 对于子代2,从P1的中间部分(4, 5, 6, 7)替换到子代2的相应位置,映射关系是:(4↔6), (5↔5), (6↔4), (7↔7)。
    \item 处理后,子代2可能变成: [9, 8, 7, 4, 5, 6, 3, 2, 1]
\end{itemize}

\section*{优点}
\begin{itemize}
    \item \textbf{保持序列信息}:PMX保留了父代个体的部分基因序列,从而有助于维持解的质量。
    \item \textbf{防止重复}:通过映射关系,PMX确保每个基因在子代中只出现一次,适用于排列问题。
\end{itemize}

\section*{缺点}
\begin{itemize}
    \item \textbf{复杂性}:由于需要处理映射关系,PMX的实现相对复杂。
    \item \textbf{局限性}:适用于排列编码问题,对于其他编码类型的适用性较低。
\end{itemize}

\section*{结论}
总的来说,部分映射交叉(PMX)是遗传算法中一种有效的交叉操作方法,特别适用于需要保持基因序列和唯一性的组合优化问题。

\end{document}
