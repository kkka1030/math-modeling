\documentclass{article}
\usepackage{amsmath}

\begin{document}

\section*{单目标优化}

单目标优化(Single Objective Optimization)是指在优化过程中只涉及一个目标函数的优化问题。其核心在于找到一个变量或一组变量的值,使得该目标函数达到最优(最大或最小)。单目标优化在工程、经济、管理等各个领域中有广泛应用。

\subsection*{单目标优化的基本概念}

\begin{itemize}
    \item \textbf{目标函数}:需要优化的函数,通常记为 $ f(x) $,其中 $ x $ 是决策变量或设计变量的向量。
    \item \textbf{决策变量}:优化过程中可以调整的变量,通常表示为向量 $ x $。
    \item \textbf{约束条件}:决策变量必须满足的一些限制条件,可以分为等式约束和不等式约束。一般形式为 $ g_i(x) \leq 0 $ 和 $ h_j(x) = 0 $。
    \item \textbf{可行域}:所有满足约束条件的决策变量组成的集合,即决策变量的取值范围。
    \item \textbf{最优解}:使目标函数达到最优(最大或最小)的决策变量的值。
\end{itemize}

\subsection*{单目标优化的分类}

根据目标函数和约束条件的不同,单目标优化问题可以分为以下几类:

\begin{itemize}
    \item \textbf{无约束优化}:目标函数没有约束条件,优化问题可以用如下形式表示:
    \[
    \text{minimize} \; f(x)
    \]
    \item \textbf{有约束优化}:目标函数有约束条件,优化问题可以用如下形式表示:
    \[
    \begin{aligned}
    & \text{minimize} \; f(x) \\
    & \text{subject to} \; g_i(x) \leq 0, \; i = 1, 2, \ldots, m \\
    & h_j(x) = 0, \; j = 1, 2, \ldots, p
    \end{aligned}
    \]
\end{itemize}

\subsection*{常见的单目标优化方法}

\begin{itemize}
    \item \textbf{解析法}:适用于简单的、可导的目标函数,通过求导数并求解方程得到最优解。
    \item \textbf{数值法}:适用于复杂的、不可导的目标函数,通过迭代的方法逐步逼近最优解,如梯度下降法、牛顿法等。
    \item \textbf{启发式算法}:适用于复杂的、多峰的、非连续的目标函数,通过模拟自然过程或生物行为来寻找最优解,如遗传算法、粒子群优化、模拟退火等。
    \item \textbf{线性规划}:适用于目标函数和约束条件均为线性的优化问题,通过单纯形法等方法求解。
\end{itemize}

\subsection*{单目标优化的应用}

单目标优化在实际应用中非常广泛,例如:

\begin{itemize}
    \item \textbf{工程设计}:如结构优化、参数调优等。
    \item \textbf{经济管理}:如投资组合优化、资源分配等。
    \item \textbf{机器学习}:如模型参数优化、超参数调优等。
\end{itemize}

总结来说,单目标优化是最常见的一类优化问题,重点在于如何选择合适的优化方法来解决特定的优化问题。

\end{document}
